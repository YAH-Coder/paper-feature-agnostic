\section{Introduction}
- when companies develop new software solutions they often employ a technique called clone-and-own
- thus the different software solutions are similar but different, in software product lines we call that variants
- with more and more variants it becomes increasingly hard to keep track of bugs in the different solutions
- if a bug is found in one variant it is hard to fix the same bug in a different variant
- this is where software product lines come into play, they allow to orgranize variants into different code blocks so called features
- features can be included or excluded and allow for one entry point to fix a bug in all variants
- of course when the company employed the technique of code-and-own it is hard to come from the variants to a structured and organized software product line with clean features
- this is where software product line reengineering comes into play
