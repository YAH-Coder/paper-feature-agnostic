\section{Related work}
What exists already

- ExtractorPL (Ziadi et al 2014)
  - Language-independent model (FSTs) for reverse-engineering SPL from variants
  - Needs language-specific parsers / pretty printers
  - Produces full SPL implementation and regenerates original products 
  
- FeatureHouse / CIDE / language-independent reference checking
  - Language-independent abstractions over multiple languages
  - Still depend on per-language front-ends
[1]: https://www.se.cs.uni-saarland.de/publications/docs/SC2009mod.pdf "Language-Independent Quantification and Weaving for Feature Composition"
[2]: https://www.se.cs.uni-saarland.de/publications/docs/TR-0208.pdf "Language-Independent Safe Decomposition of Legacy "

- Variability mining / feature location
  - Variability Mining (Kästner et al), LEADT: AST + type system + topology + text 
  - Wille, VarMine, etc: combine structural and text-based techniques, sometimes domain-specific 
[3]: https://www.sciencedirect.com/science/article/pii/S0167642318301382 "Improving custom-tailored variability mining using outlier "

- Clone-based and text-based analyses
  - Token-based clone detectors (CCFinder, SourcererCC) widely used; sometimes integrated into SPL migration
[4]: https://www.uni-bremen.de/fileadmin/user_upload/fachbereiche/fb3/softtech/Abschlussarbeiten/Diplomarbeiten/Inkrementelle_Klonerkennung.pdf "Incremental Clone Detection"

