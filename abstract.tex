\begin{abstract}
% context: SPL reengineering -> language-specific parsers; high manual effort
% need: language-independent analysis; feature-agnostic scope
% goal: full-circle reengineering; extract features; regenerate variants
% approach: token-based; character-class tokenization; stratified similarity (file/line/token)
% alignment: LCS at token level; isolate fine-grained blocks
% performance: pruning identical files; level-wise comparison; reduced comparisons
% artifacts: polyglot code; configs; build scripts; DSLs; textual resources
% evaluation: synthetic SPL; ArgoUML benchmark
% contributions: pipeline; prototype; regeneration capability; reduced reengineering effort
% limitations: assumes aligned paths; no rename/move tracking; textual—not semantic—equivalence
% audience: researchers; tool builders; SPL reengineering
% future: empirical validation; scaling; feature grouping and modeling
Reengineering software product lines (SPLs) from clone-and-own portfolios often relies on language-specific tooling and substantial manual effort to isolate, relate, and reconstruct features across variants. This paper presents a \emph{programming-language independent} and \emph{feature-agnostic} method that supports full-circle reengineering by recovering a unified representation from multiple variants and regenerating the originals from it. The approach treats all artefacts uniformly as text, covering polyglot source code, configuration files, build scripts, domain-specific languages, and other textual resources. Artefacts are tokenized along character-class boundaries and processed by a stratified similarity pipeline that collapses identical files, identifies matching and non-matching lines, and applies token-level alignment via the Longest Common Subsequence to obtain fine-grained blocks characteristic of commonalities and differences. Similarity-level pruning reduces redundant comparisons and keeps the analysis tractable for small to medium-sized variant sets. A prototype implementation demonstrates feasibility in a non-trivial synthetic SPL and the ArgoUML benchmark, showing that the recovered representation is precise enough to regenerate variants while avoiding dependence on parsers, abstract syntax trees, or pretty printers. The work contributes a parser-free extraction pipeline, a proof-of-concept tool, and initial empirical evidence that programming-language independent text processing can support SPL reengineering across heterogeneous artefacts. We spell out current assumptions (aligned paths, text-level equivalence) and outline future steps, including validation on SPLs from industrial practice and extensions towards additional resources such as string tables, help files, and other non-code artefacts.
\end{abstract}
